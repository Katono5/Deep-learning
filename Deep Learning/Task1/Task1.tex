\begin{figure}
\centering
\includegraphics{https://s3.bmp.ovh/imgs/2023/02/06/1b0360e83abd08fe.png}
\caption{Structure}
\end{figure}

\hypertarget{ux6846ux67b6ux6a21ux578b}{%
\subsection{框架模型}\label{ux6846ux67b6ux6a21ux578b}}

为了方便看结构我把每个功能单列为一个python文件放在了Structure1文件夹下 -
custom\_model:对本题网络的搭建实现,计算flop和参数量(已被注释) -
datasets:读取数据集的方法,对数据集的通道,大小进行调整统一,实现了实验数据集和验证数据集的分离
- params:超参数控制台,存放了所有超参数便于后续调参,也存放了文件路径 -
run:训练模型的主体代码 -
train\_valid\_split:单独的一块代码,在开始run之前要单独运行一次,目的是把一个完整的数据集划分成一个train和一个valid
- log:存放log模型和最后画图的function -
output文件夹:存放实验中保存的训练模型和训练日志

\hypertarget{train-valid-split}{%
\subsubsection{Train-valid split}\label{train-valid-split}}

本题数据集没有划分train和valid,所以需要自己划分,这个程序就是从原本的数据集中随机抽80\%作为训练集,剩下20\%作为验证集,单独写出来这一个程序是为了方便后边Task2,Task3都可以用。

\hypertarget{custom-model}{%
\subsubsection{Custom-Model}\label{custom-model}}

本体已经给出了网络的图,按着图一行一行打就行,为了避免多次重复操作,残差的那一个小block我单独拎出来写了一个class,在搭建完整网络的时候直接插入就好,在这个网络后边有一个被注释掉的小块代码

\begin{figure}
\centering
\includegraphics{https://s3.bmp.ovh/imgs/2023/02/06/8f675c8cec2f6773.png}
\caption{Counting params and flops}
\end{figure}

是\textbf{计算网络flop和params}用的,这里我调了thop包,为了保证代码运行过程连贯这一步的结果我单独放在这个报告里。

\begin{figure}
\centering
\includegraphics{https://s3.bmp.ovh/imgs/2023/02/06/5ed98833917b0a59.png}
\caption{output}
\end{figure}

\hypertarget{datasets}{%
\subsubsection{Datasets}\label{datasets}}

这一个文件我写的有点长了,操作有些地方也比较不清晰,但原理就是对图片进行统一resize到112x112,如果不是RGB三通道,那就调整到三通道

\begin{figure}
\centering
\includegraphics{https://s3.bmp.ovh/imgs/2023/02/06/df948e9ee674cbda.png}
\caption{Pic Resize}
\end{figure}

\begin{figure}
\centering
\includegraphics{https://s3.bmp.ovh/imgs/2023/02/06/ac526bfff2758135.png}
\caption{Convert Channels}
\end{figure}

然后一个一个把文件和label从文件夹里读出来放在files和labels列表里方便后续读取,同时把Train和Valid数据集分开读取。

\hypertarget{log}{%
\subsubsection{Log}\label{log}}

也没啥可说的,先有一个模板,然后分别输出到日志文件(./output/log.txt)里和终端里让用户看到目前的情况,在日志函数后还有一个画图的函数,把日志文件里的数据用正则表达式一个一个读取到列表里,最后统一画图,用正则表达式来实现画图的操作是因为我觉得最后从一个txt文件里读数据再画图会快很多。

\hypertarget{run}{%
\subsubsection{Run}\label{run}}

训练模型的主体部分,把数据读取出来,清空一下日志文件(因为写的时候我是用append方法写进去的不会自己删除),然后调用规定好的超参数开始训练,把网络,优化器,损失函数都规定好以后就可以扔DEVICE里开始训练了。

\hypertarget{ux653eux4e00ux4e0bux7ed3ux679cux56fe}{%
\subsection{放一下结果图}\label{ux653eux4e00ux4e0bux7ed3ux679cux56fe}}

\begin{figure}
\centering
\includegraphics{https://s3.bmp.ovh/imgs/2023/02/06/79576a1128a7b943.jpg}
\caption{Results}
\end{figure}
